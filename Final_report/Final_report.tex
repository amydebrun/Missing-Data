\documentclass{article}

\usepackage{arxiv}

\usepackage[utf8]{inputenc} % allow utf-8 input
\usepackage[T1]{fontenc}    % use 8-bit T1 fonts
\usepackage{lmodern}        % https://github.com/rstudio/rticles/issues/343
\usepackage{hyperref}       % hyperlinks
\usepackage{url}            % simple URL typesetting
\usepackage{booktabs}       % professional-quality tables
\usepackage{amsfonts}       % blackboard math symbols
\usepackage{nicefrac}       % compact symbols for 1/2, etc.
\usepackage{microtype}      % microtypography
\usepackage{graphicx}

\title{Estimating Treatment Effects in Longitudinal Clinical Trials with
Missing Data}

\author{
    Amy Browne
   \\
    Department of Mathametics \\
    University of Galway \\
   \\
  \texttt{\href{mailto:k.yip2@universityofgalway.ie}{\nolinkurl{k.yip2@universityofgalway.ie}}} \\
   \And
    Tsz Mang Yip
   \\
    Department of Mathametics \\
    University of Galway \\
   \\
  \texttt{\href{mailto:a.browne47@universityofgalway.ie}{\nolinkurl{a.browne47@universityofgalway.ie}}} \\
  }


% tightlist command for lists without linebreak
\providecommand{\tightlist}{%
  \setlength{\itemsep}{0pt}\setlength{\parskip}{0pt}}



\newcommand{\pandocbounded}[1]{#1}
\begin{document}
\maketitle


\begin{abstract}
Enter the text of your abstract here.
\end{abstract}

\keywords{
    Missing Data
   \and
    Longitudinal Data
   \and
    Randomized Clinical Trial
   \and
    Real World Data Analysis
  }

\section{Introduction}\label{introduction}

\section{Project Aims}\label{project-aims}

\section{Background}\label{background}

Missing data in clinical research

Missing data mechanisms

Missing data mechanisms are important to consider when choosing which
sort of missing data handling method to use. There are three mechanisms
which missing data can follow:

\begin{itemize}
\tightlist
\item
  Missing Completely At Random (MCAR)
\item
  Missing At Random (MAR)
\item
  Missing Not At Random (MNAR)
\end{itemize}

Although they may appear similar at first glance, handling missing data
without considering these mechanisms may result in biased estimates and
inaccurate conclusions.

\subsection{Missing Completely At
Random}\label{missing-completely-at-random}

\subsection{Missing At Random}\label{missing-at-random}

\subsection{Missing Not At Random}\label{missing-not-at-random}

\section{Data}\label{data}

\subsection{Acupuncture for chronic headache in primary care: large,
pragmatic, randomised trial (Vickers et al.,
2004)}\label{acupuncture-for-chronic-headache-in-primary-care-large-pragmatic-randomised-trial-vickers-et-al.-2004}

\subsection{The Effects of Vitamin D and Marine Omega-3 Fatty Acid
Supplementation on Chronic Knee Pain in Older US Adults: Results From a
Randomized Trial (MacFarlane et
al.~2020)}\label{the-effects-of-vitamin-d-and-marine-omega-3-fatty-acid-supplementation-on-chronic-knee-pain-in-older-us-adults-results-from-a-randomized-trial-macfarlane-et-al.-2020}

\section{Methods}\label{methods}

\subsection{Complete Case Analysis}\label{complete-case-analysis}

\subsection{Single Imputation}\label{single-imputation}

\subsection{Maximum Likelihood}\label{maximum-likelihood}

\subsection{Multiple Imputation}\label{multiple-imputation}

\section{Result}\label{result}

\section{Discussion}\label{discussion}

\bibliographystyle{unsrt}
\bibliography{references.bib}


\end{document}
