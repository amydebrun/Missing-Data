\documentclass{article}

\usepackage{arxiv}

\usepackage[utf8]{inputenc} % allow utf-8 input
\usepackage[T1]{fontenc}    % use 8-bit T1 fonts
\usepackage{lmodern}        % https://github.com/rstudio/rticles/issues/343
\usepackage{hyperref}       % hyperlinks
\usepackage{url}            % simple URL typesetting
\usepackage{booktabs}       % professional-quality tables
\usepackage{amsfonts}       % blackboard math symbols
\usepackage{nicefrac}       % compact symbols for 1/2, etc.
\usepackage{microtype}      % microtypography
\usepackage{graphicx}

\title{Estimating Treatment Effects in Longitudinal Clinical Trials with
Missing Data}

\author{
    Amy Browne
   \\
    Department of Mathametics \\
    University of Galway \\
   \\
  \texttt{\href{mailto:k.yip2@universityofgalway.ie}{\nolinkurl{k.yip2@universityofgalway.ie}}} \\
   \And
    Tsz Mang Yip
   \\
    Department of Mathametics \\
    University of Galway \\
   \\
  \texttt{\href{mailto:a.browne47@universityofgalway.ie}{\nolinkurl{a.browne47@universityofgalway.ie}}} \\
  }


% tightlist command for lists without linebreak
\providecommand{\tightlist}{%
  \setlength{\itemsep}{0pt}\setlength{\parskip}{0pt}}



\newcommand{\pandocbounded}[1]{#1}
\begin{document}
\maketitle


\begin{abstract}
Enter the text of your abstract here.
\end{abstract}

\keywords{
    Missing Data
   \and
    Longitudinal Data
   \and
    Randomized Clinical Trial
   \and
    Real World Data Analysis
  }

\section{Introduction}\label{introduction}

\section{Project Aims}\label{project-aims}

\section{Background}\label{background}

Missing data in clinical research

\subsection{Missing Data Mechanisms}\label{missing-data-mechanisms}

Missing data mechanisms are important to consider when choosing which
sort of missing data handling method to use. There are three mechanisms
which missing data can follow:

\begin{itemize}
\tightlist
\item
  Missing Completely At Random (MCAR)
\item
  Missing At Random (MAR)
\item
  Missing Not At Random (MNAR)
\end{itemize}

Although they may appear similar at first glance, handling missing data
without considering these mechanisms may result in biased estimates and
inaccurate conclusions.

\textbf{Missing Completely At Random}

The formal definition of MCAR data is:

\[\quad P(R = 1 \mid Y, X) = P(R = 1)\] where \(R\) is a missing data
indicator (1 = Data is observed, 0 = Data is missing), \(Y\) represents
the variables in which the data is potentially missing and \(X\)
represents the observed data. The probability of the data is observed
given observed data and missing data is the same as the probability of
being observed without the given data. This mechanism is considered the
easiest to deal with as it does not bias the result although data is
rarely MCAR. This can occur due to system failure and some data is
deleted accidentally, or else there is issues with the treatment system
and data cannot be recorded.

\textbf{Missing At Random}

\[\quad P(R = 1 \mid Y, X) = P(R = 1 \mid X)\] The probability of data
being observed given the rest of the data is the same as the probability
being observed given the observed data. In short, the data's missingness
is dependent on the observed data. For example, people with a higher
body mass index may be more prone to having missing blood pressure data
- this is not relative to the missing data. MAR is a more realistic
mechanism than MCAR and requires more intensive handling methods.

\textbf{Missing Not At Random}

\[\quad P(R = 1 \mid Y, X) \ne P(R = 1 \mid X)\] The probability of data
being observed is not dependent on the observed data. This mechanism is
the most difficult to deal with as it relates to the unobserved data, so
producing valid results is a challenge. Certain participants in a
general health study may avoid answering questions truthfully about
smoking habits or their diet in order to make themselves more appealing.
Sensitivity analysis is an option to determine the treatment effect when
assuming different mechanisms.

\section{Data}\label{data}

\subsection{Acupuncture for chronic headache in primary care: large,
pragmatic, randomised trial (Vickers et al.,
2004)}\label{acupuncture-for-chronic-headache-in-primary-care-large-pragmatic-randomised-trial-vickers-et-al.-2004}

\subsection{The Effects of Vitamin D and Marine Omega-3 Fatty Acid
Supplementation on Chronic Knee Pain in Older US Adults: Results From a
Randomized Trial (MacFarlane et
al.~2020)}\label{the-effects-of-vitamin-d-and-marine-omega-3-fatty-acid-supplementation-on-chronic-knee-pain-in-older-us-adults-results-from-a-randomized-trial-macfarlane-et-al.-2020}

\section{Methods}\label{methods}

\subsection{Complete Case Analysis}\label{complete-case-analysis}

\subsection{Single Imputation}\label{single-imputation}

\subsection{Maximum Likelihood}\label{maximum-likelihood}

\subsection{Multiple Imputation}\label{multiple-imputation}

\section{Result}\label{result}

\section{Discussion}\label{discussion}

\bibliographystyle{unsrt}
\bibliography{references.bib}


\end{document}
