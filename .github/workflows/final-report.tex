\documentclass[]{ametsoc}
\usepackage{color}
\usepackage{hyperref}
\journal{}
%  Please choose a journal abbreviation to use above from the following list:
%
%   jamc     (Journal of Applied Meteorology and Climatology)
%   jtech     (Journal of Atmospheric and Oceanic Technology)
%   jhm      (Journal of Hydrometeorology)
%   jpo     (Journal of Physical Oceanography)
%   jas      (Journal of Atmospheric Sciences)
%   jcli      (Journal of Climate)
%   mwr      (Monthly Weather Review)
%   wcas      (Weather, Climate, and Society)
%   waf       (Weather and Forecasting)
%   bams (Bulletin of the American Meteorological Society)
%   ei    (Earth Interactions)

%%%%%%%%%%%%%%%%%%%%%%%%%%%%%%%%
%Citations should be of the form ``author year''  not ``author, year''
\bibpunct{(}{)}{;}{a}{}{,}
\title{Report}





% tightlist command for lists without linebreak
\providecommand{\tightlist}{%
  \setlength{\itemsep}{0pt}\setlength{\parskip}{0pt}}




\abstract{}
\begin{document}
\maketitle
c The report should be structured similarly to an academic paper and
include the project aims, background, and give greater detail on the
methods results, the interpretation of results and conclusions, with an
appropriate bibliography. It should be produced using R Markdown using a
standard template (see below).

\subsection{Project Aims}\label{project-aims}

\subsection{Background}\label{background}

\subsection{Method}\label{method}

\subsection{Results}\label{results}

\subsection{Interpretation}\label{interpretation}

\subsection{Conclusions}\label{conclusions}

\subsection{Bibliography}\label{bibliography}
\end{document}
